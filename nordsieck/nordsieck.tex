\documentclass[11pt,a4paper]{article}
\usepackage[nohead]{geometry}
\usepackage{amsmath,amsfonts,physics}
\geometry{left=1in,right=1in,top=0.6in,bottom=1in}
\usepackage[capitalise,noabbrev]{cleveref}
\usepackage[bottom]{footmisc}
\usepackage{multicol}

\setlength{\columnseprule}{0.4pt}
\setcounter{footnote}{1}
\title{Nordsieck Form Implementation Note}
\author{Yingbo Ma\thanks{Email: \tt{mayingbo5@gmail.com},
                         GitHub: \tt{@YingboMa}}}
\date{April 2018}

\begin{document}
\maketitle
%\tableofcontents

\section*{Notations}
\begin{multicols}{2}
  \begin{itemize}
    \item For any variable $X$, let $X_n$ denote the $n$-th step of the variable $X$.
    \item For any variable $X$, let $\hat{X}$ denote the predicted value of the variable $X$.
    \item Let $\otimes$ denote the Kronecker product.
    \item Let $q$ denote the order of the method.
    \item Let $u$ denote a vector of dependent variables.
    \item Let $t$ denote a scalar independent variable.
    \columnbreak
    \item Let $F$ denote the vector equation for a ordinary differential equation
      $u' = F(u, t)$.
    \item Let $z_n$ denote the Nordsieck vector, which is
      \[
        z_n = \left[\frac{y_n}{0!}, h\frac{y_n'}{1!}, h^2\frac{y_n''}{2!},
        \cdots, h^q\frac{y_n^{(q)}}{q!}\right]^T.
      \]
    \item Let $A(q)$ denote a $(q+1)\times (q+1)$ Pascal triangle matrix (lower
      triangular).
    \item Let $P(q)$ denote a $(q+1)\times (q+1)$ Pascal triangle matrix (upper
      triangular).
  \end{itemize}
\end{multicols}

\section{The General Form of Nordsieck Methods}
The Adams methods are usually presented as
\begin{equation}\label{eq:Adams_int}
  u_{n+1} = u_n + \int_{t_n}^{t_{n+1}} P(\tau) \dd{\tau},
\end{equation}
where $P(t)$ is an interpolation polynomial through points $t_i, F_i$ for
$i=n-q+1,\cdots,n$. The backward differentiation formula methods are usually
presented as
\begin{equation}\label{eq:BDF_int}
  \sum_{j=1}^q j^{-1}\nabla^ju_{n+1} = hF_{n+1},
\end{equation}
where $\nabla$ is a backward difference operator\footnote{A backward
difference operator has the recurrence relation of
\[
  \nabla^0u_n = u_n,\quad \nabla^{j+1}u_n = \nabla^ju_n-\nabla^ju_{n-1}.
\]}.
However, another perspective of those multi-step numerical schemes are from
polynomials with constraints directly. That is, a polynomial with $L\equiv q+1$
constraints one can full encode ~\cref{eq:Adams_int} and~\cref{eq:BDF_int}.

\subsubsection{Adams Methods}
Explicit Adams methods can be thought as a polynomial interpolation that
fulfills
\begin{align}
  \eta'_{n-1}(t_{n-i})&=F_{n-i}\qq{where} i=1,2,\cdots,q \\
  \eta_{n-1}(t_{n-1})&=u_{n-1},
\end{align}
while the implicit Adams methods are polynomials that fulfills
\begin{align}
  \eta'_{n}(t_{n})&=F_{n}\qq{where} i=0,1,\cdots,q-1 \\
  \eta_{n}(t_{n})&=u_{n}.
\end{align}
Here, we define the predictors as a polynomial interpolation
\begin{equation}
  \hat{u}_n = \eta_{n-1}(t_n),\quad \hat{u}'_n = \eta'_{n-1}(t_n).
\end{equation}

\subsubsection{BDF Methods}

\section{Calculation of Coefficients}

\section{Equivalence with Adams and BDF methods}

\end{document}
